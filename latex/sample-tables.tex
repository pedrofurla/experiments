% !TEX TS-program = pdfLaTeX
% !TEX encoding = UTF-8 Unicode

\documentclass[12pt]{article}  % tipo de documento

\usepackage[utf8]{inputenc}  % levar em consideração que é um arquivo em utf8
\usepackage{times}                 % um fonte que tenha acentos
\usepackage[T1]{fontenc}      % levando em consideração utf8 na hora de escolher as letras da fonte
\usepackage{hyperref}
\usepackage{fancyhdr}           % manipular o cabeçalho
\usepackage[normalem]{ulem} % sublinhado que não sobreescreve \emph
\usepackage{multirow}

\pagestyle{fancy}
\setcounter{tocdepth}{5}

\rhead{Deploy do Pelotão no espaço}

\begin{document} 

\title{\LARGE Tables in \LaTeX }
%\maketitle

\section{TESTES} 

\subsection{Tabela1}

\begin{tabular}{|  l  | c | r | }
  \hline                       
  1 & 2 & 3 \\ \hline
  4 & 5 & 6 \\ \hline
  7 & 8 & 9 \\
  \hline  
\end{tabular}

\subsection{Column Span} 

\begin{tabular}{|  l  | c | r | }
  \hline                       
  1 & 2 & 3 \\ \hline
  \multicolumn{2}{|c|}{ 5 } & 6 \\ \hline
  7 & 8 & 9 \\ 
  \hline  
\end{tabular}



\begin{tabular}{|  l  | c | r | }
  \hline      
  \multirow{3}{*}{1} & 2 & 3 \\ \cline{2-3}
   & 5 & 6 \\ \cline{2-3}
   & 8 & 9 \\
  \hline  
\end{tabular}

\subsection{Lists}

\subsubsection{Bullets}


\subsubsection{Enumerados}

\begin{enumerate}
  \item The first item
  \item The second item
  \item The third etc \ldots
\end{enumerate}


\end{document}

