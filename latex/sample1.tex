% !TEX TS-program = pdfLaTeX
% !TEX encoding = UTF-8

% 

\documentclass[12pt]{article}  % tipo de documento

\usepackage[utf8]{inputenc}  % levar em consideração que é um arquivo em utf8
\usepackage{times}                 % um fonte que tenha acentos
\usepackage[T1]{fontenc}      % levando em consideração utf8 na hora de escolher as letras da fonte
\usepackage{hyperref}
\usepackage{fancyhdr}           % manipular o cabeçalho
\usepackage[normalem]{ulem} % sublinhado que não sobreescreve \emph

\pagestyle{fancy}
\setcounter{tocdepth}{5}

\rhead{Deploy do Pelotão no espaço}

\begin{document} 

\newcommand{\pelota}[1]{\uwave{#1}}
\newcommand{\bleh}{Exemplo de substituição}


\title{\LARGE 
	Le petit example de \LaTeX \\ 
	avec accents
}
\maketitle

\tableofcontents

\section{TESTES} 

áéíóú \\
âêîôû \\
ãeiõu \\
aëïöü \\
àèìòù

Uma parâgrafo grande para mostrar um pouco mais de como o \LaTeX\ renderiza as coisas. Ao contrário de ferramentas como MS Word, que são \emph{WYSIWYG} ``What you see is what you get'', \LaTeX\ é \textit{WYSIWYM} ``What you see is what you mean''. Mais algumas letras quase aleatórias, seriam realmente aleatórias não fosse por formarem palavras.

\pelota{Exemplo de comando customizado}

\bleh

\subsection{Tabela1}

\begin{tabular}{|  l  | c | r | }
  \hline                       
  1 & 2 & 3 \\ \hline
  4 & 5 & 6 \\ \hline
  7 & 8 & 9 \\
  \hline  
\end{tabular}

\subsection{Tabela2} 

\begin{LARGE}
\begin{tabular}{|  l  | c | r | }
  \hline                       
  1 & 2 & 3 \\ \hline
  4 & 5 & 6 \\ \hline
  7 & 8 & 9 \\ 
  \hline  
\end{tabular}
\end{LARGE}

\subsection{Lists}

\subsubsection{Bullets}

\begin{itemize}
  \item italics \textit{The first item}
  \item bold \textbf{The second item}
  \item underline \uline{The third etc \ldots}
  \item underwave \uwave{More stuff}
  \item strike-out \sout{Even more ...}
\end{itemize}

\subsubsection{Enumerados}

\begin{enumerate}
  \item The first item
  \item The second item
  \item The third etc \ldots
\end{enumerate}


\end{document}











