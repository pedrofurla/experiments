% !TEX TS-program = pdfLaTeX
% !TEX encoding = UTF-8 Unicode
% !TEX spellcheck = fr_FR

\documentclass[12pt]{article}  % tipo de documento

\usepackage[utf8]{inputenc}  % levar em consideração que é um arquivo em utf8
\usepackage{times}                 % um fonte que tenha acentos
\usepackage[T1]{fontenc}      % levando em consideração utf8 na hora de escolher as letras da fonte
\usepackage{hyperref}
\usepackage[normalem]{ulem} % sublinhado que não sobreescreve \emph
\usepackage{multirow}
\usepackage{amsmath}
\usepackage{amsfonts}
\usepackage{tabularx}
\usepackage[xcdraw,table]{xcolor}
\usepackage[margin=1.5cm]{geometry}

\begin{document} 

\newcommand{\bo}[1]{\textbf{#1}}
%\newcommand{\it}[1]{\textit{#1}}
\newcommand{\rarrow}{\begin{math}\to\ \end{math}}


{\LARGE Mes études du Français en \LaTeX}

\section{Articles} 

\subsection{Indéfini}

\begin{tabular}{| c | c | c | l |}
  \hline                       
  \multirow{2}{*}{singulier} 
  	& masculin      & un & \bo{un} fils  \\ \cline{2-4}
  	& féminin       & une & \bo{une} fille \\ \cline{2-4}
  \multirow{3}{*}{pluriel} 
  	& & \multirow{3}{*}{des} & \bo{des} fils \\ \cline{4-4}
    & &  & \bo{des} filles \\ \cline{4-4}
    & &  & \bo{des} enfants \\
  \hline  
\end{tabular}

\begin{itemize}
  \item \bo{un} boulanger, \bo{une} boulangère 
  \item \bo{des} boulangers, \bo{des} boulangères
  \item \bo{un} cahier, \bo{un} ami, \bo{un} hôtel
  \item \bo{des} cahiers, \bo{des} amis, \bo{des} hôtels
  \item \bo{une} église, \bo{une} amie, \bo{une} maison
  \item \bo{des} églises, \bo{des} amies, \bo{des} maisons
\end{itemize}

\subsection{Défini}

\begin{tabular}{| c | c | c | l |}
  \hline                       
  \multirow{3}{*}{singulier} 
  	& masculin      & le & \bo{le} fils  \\ \cline{2-4}
  	& féminin       & la & \bo{la} fille \\ \cline{2-4}
  	& avant voyelle & l' & \bo{l'}enfant \\ \hline
  \multirow{3}{*}{pluriel} 
  	& & \multirow{3}{*}{les} & \bo{les} fils \\ \cline{4-4}
    & &  & \bo{les} filles \\ \cline{4-4}
    & &  & \bo{les} enfants \\
  \hline  
\end{tabular}

\subsection{Articles définis avec les prépositions \it{à} et \it{de}} 

%\begin{tabular}{|p{3.5cm}| l | }
\begin{tabular}{| l | l | }
  \hline                       
  \multirow{2}{*}{masculin singulier} 
  	& \bo{de} + \bo{le} \rarrow \bo{du}  \\ \cline{2-2}
    & \bo{à}  + \bo{le} \rarrow \bo{au}  \\ \hline
  \multirow{4}{*}{féminin singulier ou avant une voyelle} 
  	& \bo{de} + \bo{la} \rarrow \bo{de la}  \\ \cline{2-2}
  	& \bo{de} + \bo{l'} \rarrow \bo{de l'}  \\ \cline{2-2}
    & \bo{à}  + \bo{la} \rarrow \bo{á la} \\ \cline{2-2}
    & \bo{à}  + \bo{l'} \rarrow \bo{á l'}  \\ \hline
  \multirow{2}{*}{masculin et féminin pluriel} 
  	& \bo{de} + \bo{les} \rarrow \bo{des}  \\ \cline{2-2}
    & \bo{à}  + \bo{les} \rarrow \bo{aux} \\ \hline
\end{tabular}

\subsection{Les démonstratifs}

\begin{tabular}{| l | c | l |}
  \hline                       
  \multirow{2}{*}{masculin singulier} 
  	& ce & \bo{Ce} jardin  \\ \cline{3-3}
    &    & \bo{Ce} hôtel   \\ \hline
  \multirow{2}{*}{féminin singulier} 
  	& cette & \bo{Cette} maison  \\ \cline{3-3}
  	&       & \bo{Cette} rose    \\ \hline 	
  \multirow{2}{*}{singulier avant une voyelle} 
  	& cet & \bo{Cet} homme  \\ \cline{3-3}
  	&     & \bo{Cet} actrice  \\ \hline
  \multirow{2}{*}{pluriel} 
  	& ces & \bo{Ces} hôtels  \\ \cline{3-3}
    &     & \bo{Ces} actrices \\ \hline
\end{tabular}

\section{Pronoms}

\newcommand{\mc}[3]{\multicolumn{#1}{#2}{#3}}

\begin{tabular}{| l | l | l | l | l | l |}
  \hline
            &            & \mc{3}{|c|}{possessif} \\ \hline
  personnel & complement & \mc{2}{|c|}{singulier} & pluriel \\ \hline
            &            & masculin  & feminin    &         \\ \hline
  je        & \bo{moi}   & \bo{mon}  &\bo{ma}     &\bo{mes} \\ \hline
  tu        & \bo{toi}   & \bo{ton}  &\bo{ta}     &\bo{tes} \\ \hline
  il        & \bo{lui}   & \bo{sa}   &\bo{sa}     &\bo{ses} \\ \hline
  elle      & \bo{elle}  & \bo{sa}   &\bo{sa}     &\bo{ses} \\ \hline
  nous      & \bo{nous}  & \bo{notre}&\bo{notre}  &\bo{nos} \\ \hline
  vous      & \bo{vous}  & \bo{votre}&\bo{votre}  &\bo{vos} \\ \hline
  ils       & \bo{eux}   & \bo{leur} &\bo{leur}   &\bo{leurs} \\ \hline
  elles     & \bo{elles} & \bo{leur} &\bo{leur}   &\bo{leurs} \\ 
  \hline  
\end{tabular}

\section{Verbes}

\newcommand{\verbs}[1]{
\rowcolors*{1}{gray!45}{gray!45}
\begin{tabular}{| l | c | c | c | c | c | c | c |}
  \showrowcolors 
  \hline
  \bo{infinitif} & \bo{participe} & \bo{je} & \bo{tu} & \bo{il} & \bo{nous} & \bo{vous} & \bo{ils} \\ 
  & \bo{passé} & & & \bo{elle} & & & \bo{elles} \\ 
  &  & & & \bo{on} & & & \\ \hline
  \hiderowcolors
#1
\end{tabular}
}

%\bo{}& {\it{a.}} & & & & & &  \\ \hline

\subsection{Premier groupe}

\verbs{
\bo{appeler}& {\it{a.}} appelé& appelle& appelles& appelle& appelons& appelez& appellent  \\ \hline
\bo{manger}&  {\it{a.}} mangé&  mange& manges& mange& mangeons& mangez& mangent \\ \hline
\bo{parler}&  {\it{a.}} parlé&  parle& parles& parle& parlons& parlez& parlent \\ \hline
\bo{travailler}& {\it{a.}} & & & & & &  \\ \hline
\bo{habiter}& {\it{a.}} & & & & & &  \\ \hline
\bo{arriver}& {\it{e.}} arrivé& & & & & &  \\ \hline
\bo{}& {\it{a.}} & & & & & &  \\ \hline
}

\subsection{Deuxième groupe}

\verbs{
\bo{finir}& {\it{a.}} fini& finis& finis& finit& finissons& finissez& finissent \\ \hline
}

\subsection{Troisième groupe}

\verbs{
\bo{aller}& {\it{e.}} allé& vais& vas& va& allon& allez& vont \\ \hline
\bo{avoir}& {\it{a.}} eu& ai& as& a& avons& avez& ont  \\ \hline
\bo{être}& {\it{a.}} été& suis& es& est& sommes& êtes& sont \\ \hline
\bo{devoir}& {\it{a.}} dû& dois& dois& doit& devons& devez& doivent \\ \hline
\bo{pouvoir}& {\it{a.}} pu& peux& peux& peut& pouvons& pouvez& peuvent \\ \hline
\bo{vouloir}& {\it{a.}} voulu& veux& veux& veut& voulons& voulez& veulent \\ \hline
\bo{savoir}& {\it{a.}} su& sais& sais& sait& savons& savez& savent \\ \hline
\bo{partir}& {\it{e.}} partie& pars& pars& part& partons& partez& partent \\ \hline
\bo{faire}& {\it{a.}} fait& fais& fais& fait& faisons& faites& font \\ \hline
\bo{comprendre}& {\it{a.}} compris& comprends& comprends& comprend& comprenons& comprenez& comprennent \\ \hline
\bo{connaître}& {\it{a.}} & & & & & &  \\ \hline
\bo{lire}& {\it{a.}} & & & & & &  \\ \hline
\bo{écrire}& {\it{a.}} & & & & & &  \\ \hline
\bo{venir}& {\it{e.}} venu& viens& viens& vient& venons& venez& viennent  \\ \hline
\bo{prendre}& {\it{a.}} mis& prends& prends& prend& prenons& prenez& prennent \\ \hline
\bo{mettre}& {\it{a.}} pris& mets& mets& met& mettons& mettez& mettent\\ \hline
\bo{prendre}& {\it{a.}} & & & & & &  \\ \hline
\bo{}& {\it{a.}} & & & & & &  \\ \hline
\bo{}& {\it{a.}} & & & & & &  \\ \hline
\bo{}& {\it{a.}} & & & & & &  \\ \hline
\bo{}& {\it{a.}} & & & & & &  \\ \hline
\bo{}& {\it{a.}} & & & & & &  \\ \hline
}
\newpage

\section{Les dates et l'heures}

\begin{itemize}
  \item Elle est née le 3 avril 1960  
  	\subitem Le 3, en avril, en 1960
  \item Le mois de l'année:   
  	\subitem janvier, février, mars, avril, mai, juin, juillet, août, septembre, octobre, novembre, décembre
  \item Le jours de la semaine:   
  	\subitem lundi, mardi, mercredi, jeudi, vendredi, samedi, dimanche
  \item Hier, aujourd'hui, demain
  \subitem{
\begin{tabular}{| c  c |}
  \hline                       
    \multicolumn{2}{|c|}{la semaine dernière} \\
    \multicolumn{2}{|c|}{\begin{math}\uparrow\end{math}} \\
   15 mai & avant-hier    \\
   16 mai & hier          \\
   17 mai & aujourd'hui   \\
   18 mai & demain        \\
   19 mai & après-demain  \\
   \multicolumn{2}{|c|}{ \begin{math}\downarrow\end{math}} \\
   \multicolumn{2}{|c|}{la semaine prochaine} \\
  \hline  
\end{tabular}
} 
  \item L'heure
    \subitem Quelle heure il-est? Il est quelle heure?
    FALTA
  \item Être arrivé
  	\subitem en avance
  	\subitem à l'heure
  	\subitem en retard
\end{itemize}

\section{Forme interrogatifs}

\begin{itemize}
  \item \bo{Est-ce que}
  \item \bo{Qu'est que c'est}
  \subitem
\end{itemize}

\begin{enumerate}
  \item The first item
  \item The second item
  \item The third etc \ldots
\end{enumerate}

\end{document}

