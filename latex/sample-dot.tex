% !TEX TS-program = pdfLaTeX
% !TEX encoding = UTF-8

% 

\documentclass[12pt]{article}  % tipo de documento

\usepackage[utf8]{inputenc}  % levar em consideração que é um arquivo em utf8
\usepackage{times}                 % uma fonte que tenha acentos
\usepackage[T1]{fontenc}      % levando em consideração utf8 na hora de escolher as letras da fonte 
\usepackage{hyperref}
\usepackage{fancyhdr}           % manipular o cabeçalho
\usepackage[normalem]{ulem} % sublinhado que não sobreescreve \emph

% Ligando o suporte a arquivos dot
% lembre-se de colocar --enable-write18 na linha de comando
% o executável do graphviz precisa estar no PATH
\usepackage[pdftex]{graphicx}
\usepackage{graphvizzz} % disponível em https://code.google.com/p/graphvizzz/

% outras alternativas para colocar graphviz em latex:
% http://mark.aufflick.com/blog/2007/03/25/embedding-graphviz-in-latex-documents
% https://github.com/mprentice/GraphViz-sty
% http://www.duocoding.nl/blog/249/how-to-embed-dot-graphs-graphviz-in-latex
% http://www.dont-panic.cc/erik/2007/11/latex-and-graphviz/
% http://www.fauskes.net/nb/introducing-dot2texi/ <-- this one deserves special attention!!!

\pagestyle{fancy}
\setcounter{tocdepth}{5}

\rhead{Dot files are easily embedded}

\begin{document} 

\title{\LARGE 
	Embedding dot files into \LaTeX
}
\maketitle

\tableofcontents

\section{TESTES}

%% Start of dot diagram
\digraph
[scale=0.7]{g1}
{
   margin="0 0 0 0";
   rankdir="TD";
   node [shape=box];
   indexador [label=" Indexador Backend"];
   index [label="Índice"];
   cat [label="Catalogo"];
   tools [label="Tools"];
   sm [label="Site Manager"];
   mailing [label="Mailing"];   
   indexador -> index
   sm -> cat [label="xml disco"]
   cat -> sm [label="jms"]
   index -> cat
   sm -> index
   tools -> indexador [label="db/jms"]
   indexador -> mailing  [label="jms"]
}
%% End of dot diagram

\subsection{Tabela1}

\begin{tabular}{|  l  | c | r | }
  \hline                       
  1 & 2 & 3 \\ \hline
  4 & 5 & 6 \\ \hline
  7 & 8 & 9 \\
  \hline  
\end{tabular}

\subsection{Tabela2} 

\subsection{Lists}

\subsubsection{Bullets}

\subsubsection{Enumerados}

\end{document}











